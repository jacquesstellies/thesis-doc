% \graphicspath{{introduction/fig/}}

\chapter{Literature Review}
\label{chap:literature_review}

Here lies the literature review. Those who wander further beware.

\section{Satellite Attitude Kinematics}

The attitude of a vehicle refers to it's orientation and angular rate in 3D space. In order to control a satellite's
oritentation for pointing and other applications it's phyics must be understood and modelled. 
A satellite's attitude with mass moment of inertia $\vec{J}$ can be modelled using the Newton-Euler equation: 

\begin{equation}
\vec{J}\vec{\dot{\omega}} = \sum{\vec{N_{ext}}} -\vec{H}\times\vec{\omega}
\end{equation}
where $\vec{\omega}$ is the satellite's angular rate vector, $\vec{N_{ext}}$ is any external torque applied and $\vec{H}$
the satellite's total angular momentum. This equation is a useful 

\subsection{Attitude representations}
The satellite's orientation can be represented intuitively using euler angles such as yaw $\psi$, pitch $\theta$ and roll$\phi$.
However these representations are not so effective for computation in attitude control as they require more expensive trigonometric
functions to be used and produce singularities. The most accurate way to represent the attitude of an object is using the direction
cosine matrix (DCM) aka. the rotation matrix. This is a 3x3 matrix where each column represents the orthogonal xyz axes vector. However
this is cumbersome for computing calculations as it would require computing 9 differential equations to determine a satellite's 
attitude. Instead quaternions are used during computations. The quaternion can be defined as:

\begin{align}
    q_{1}=e_{1}\sin(\Phi) \\
    q_{2}=e_{2}\sin(\Phi) \\
    q_{3}=e_{3}\sin(\Phi) \\
    q_{4}=q_{4}\cos(\Phi) 
\end{align}



\section{Attitude Control of a Satellite}
\subsection{Dynamics and Kinematics}
Dynamics

\begin{equation}
\mathbf{J}\mathbf{\dot{\omega}} = \sum{\mathbf{T_{ext}}} -\mathbf{\omega}\times(\mathbf{J}\mathbf{\omega}-\vec{h})-\dot{\vec{h}}
\end{equation}

Kinematics
\begin{equation}
\vec{\dot{q}}=\frac{1}{2}
\begin{bmatrix}
q_{4} \ \ \neg q_{3} \ \ \ q_{2} \\ 
q_{3} \ \ q_{4} \ \ \ \neg q_{1} \\ 
\neg q_{2} \ \ q_{1} \ \ \ q_{4} \\ 
\neg q_{1} \ \ \neg q_{2} \ \ \ \neg q_{3} \\
\end{bmatrix} \vec{\omega}
\end{equation}
\subsection{}
\section{Fault Tolerant Control}

\section{FTC Applied to Satellites}
In \cite{JiangYe2010Abfc} a backstepping sliding-mode adaptive controller is designed to handle reaction wheel faults where a 
redundant actuator is available. unknown faults of the reaction wheels, external disturbances, and unknown
inertia matrix of the spacecraft. 
It is assumed that the angular velocity Ω of the reaction wheels are within the saturation limit and the input control torque 
u(t) is unbounded.

\section{Backstepping}
\cite{khalilNonlinearSystems2002} describes backstepping as a recursive design method for nonlinear systems.
\cite{ShenQiang2019AFCS} uses backstepping to design a fault-tolerant controller for a satellite with reaction wheel faults.